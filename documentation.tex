\documentclass[12pt,ebook,oneside]{book}

\usepackage{graphicx} 
\usepackage{layouts} 
\usepackage{hyperref} 
\textheight=1.2\textheight

\begin{document}

\section{Resources}

Resources convert from Vicky RGOs. In particular, the
\texttt{crude\_oil}, \texttt{metal}, \texttt{rare\_materials}, and
\texttt{energy} fields of \texttt{config.txt} regulate how much weight
each Vicky resource has for the eponymous HoI resource; each RGO then
has this weight (if not listed, it is zero) times its
\texttt{last\_income} field. 

\section{Manpower and leadership}

All POPs listed in the \texttt{fightingClasses} object have a
redistribution weight for manpower equal to their size, \emph{unless}
they work in an RGO type listed in the \texttt{manpower} object, in
which case their weight is calculated as for a resource. Notice that
by default the \texttt{manpower} object contains RGOs that have
nonzero weights for resource, and the weights in it are all zero. The
effect is that labourers who work in resource-giving RGOs do not give
manpower. 

Leadership is redistributed according to the size of the POP types
listed in the \texttt{officerClasses} object. 

\section{Industry} 

Vicky factories convert to HoI industrial capacity with a weight
proportional to their profit; the world total of IC remains what it
is in the input file. Unemployed and subsidised workers count as
making \texttt{minimumProfitRate} for weighting purposes, but the IC
they create starts damaged. Employed workers who make a positive
profit less than \texttt{minimumProfitRate} count as making it; this
means that it is never useful to close a profitable factory, though
there is some advantage to having factories that are only just barely
profitable. 

\end{document}
