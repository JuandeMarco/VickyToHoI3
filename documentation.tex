\documentclass[12pt,ebook,oneside]{book}

\usepackage{graphicx} 
\usepackage{layouts} 
\usepackage{hyperref} 
\textheight=1.2\textheight

\begin{document}

\section{Resources}

Resources convert from Vicky RGOs. In particular, the
\texttt{crude\_oil}, \texttt{metal}, \texttt{rare\_materials}, and
\texttt{energy} fields of \texttt{config.txt} regulate how much weight
each Vicky resource has for the eponymous HoI resource; each RGO then
has this weight (if not listed, it is zero) times its
\texttt{last\_income} field. 

\section{Manpower and leadership}

All POPs listed in the \texttt{fightingClasses} object have a
redistribution weight for manpower equal to their size, \emph{unless}
they work in an RGO type listed in the \texttt{manpower} object, in
which case their weight is calculated as for a resource. Notice that
by default the \texttt{manpower} object contains RGOs that have
nonzero weights for resource, and the weights in it are all zero. The
effect is that labourers who work in resource-giving RGOs do not give
manpower. 

Leadership is redistributed according to the size of the POP types
listed in the \texttt{officerClasses} object. 

\section{Industry} 

Vicky factories convert to HoI industrial capacity with a weight
proportional to their profit; the world total of IC remains what it
is in the input file. Unemployed and subsidised workers count as
making \texttt{minimumProfitRate} for weighting purposes, but the IC
they create starts damaged. Employed workers who make a positive
profit less than \texttt{minimumProfitRate} count as making it; this
means that it is never useful to close a profitable factory, though
there is some advantage to having factories that are only just barely
profitable. 

\section{Governments}

Each converted nation gets the government of the historical nation it
most closely resembles, provided no other nation resembles it even
more. That is, a resemblance is calculated for each pair of converted
and historical nations. The highest resemblance is then assigned, then
the next highest for which neither converted or historical nation has
already been used, and so on until all converted nations have a
government. For example, suppose the converted nations are SWE, DEN,
and NOR; and the historical nations are GER, ENG, and FRA. Suppose
further that the resemblances are thus:
\begin{verbatim}
SWE - GER: 10
SWE - ENG: 8
SWE - FRA: 3
DEN - GER: 9
DEN - ENG: 7
DEN - FRA: 2
NOR - GER: 2
NOR - ENG: 4
NOR - FRA: 3
\end{verbatim}
Sorting this list from highest to lowest, we get:
\begin{verbatim}
SWE - GER: 10
DEN - GER: 9
SWE - ENG: 8
DEN - ENG: 7
NOR - ENG: 4
NOR - FRA: 3
SWE - FRA: 3
DEN - FRA: 2
NOR - GER: 2
\end{verbatim}
Thus, SWE gets the historical GER government, and SWE and GER are
struck from the list, leaving:
\begin{verbatim}
DEN - ENG: 7
NOR - ENG: 4
NOR - FRA: 3
DEN - FRA: 2
\end{verbatim}
Then, DEN gets the historical ENG government and these tags are
struck, leaving only the final resemblance, from which NOR is assigned
the FRA government. 

Resemblance is calculated from the \texttt{govResemblance} object in
the configuration file. For example, consider the resemblance object
to Sweden:
\begin{verbatim}
  SWE = {
    scale = 0.5
    government = {
      fascist_dictatorship = 0
      proletarian_dictatorship = 0 
      presidential_dictatorship = 0
      bourgeois_dictatorship = 0
      absolute_monarchy = 0.1
      prussian_constitutionalism = 0.8
      hms_government = 0.5
      democracy = 0.8
    }
  }
\end{verbatim}
This says that a Victoria nation gets 0.8 resemblance points to Sweden for
having the \texttt{prussian\_constitutionalism} government, 0.5 for
\texttt{hmc\_government}, and so on. Resemblances are multiplied by
the \texttt{scale}, which is 1 by default and smaller for
historically-minor countries like Sweden; this means that a country
which equally resembles Germany and Sweden will get the German
government if it is available. In addition, human countries get a
bonus of \texttt{humanFactor} to all resemblances listed in the config
file, to advantage them
over AI minors in the scramble for interesting governments. There is
also a tiny random factor to break ties. 

Fields marked `numerical', such as plurality, create a resemblance of
their \texttt{value} key times the number in the Victoria
country. Fields with a `target' keyword look in the nested sub-object of
the Victoria nation rather than the top level. 

\section{Leaders}

Each nation, in HoI and Vicky, is considered as a point in army-navy
space; its coordinates are given by its strength relative to the
strongest nation. Thus a nation with the largest army, but a navy half
the size of the largest navy in the game, is at the point $(1, 0.5)$. 
Strength of armies is simply the sum of all regimental strengths. For
navies, it is the sum of the weights of all ships; the weights are
given in the \texttt{[hoi,vic]Ships} fields of the config object. 

Each Vicky nation gains the historical officers of a HoI nation. The
officers of the strongest HoI nation - that is, the one with the
highest sum of army and navy strength - are handed out first, and go
to the nation closest to it in army-navy space. Leaders are then
handed out in descending HoI order of army plus navy strength, always
going to the nation closest in army-navy space which has not yet
received leaders. Vicky nations with less than 0.01 combined strength
are always considered 1 unit further away than they really are, to
ensure that mid-ranked historical nations are handed to mid-ranked
Vicky nations rather than to minors with no army, which technically
may be closer - that is, (0, 0) is closer to (0.1, 0) than (0.21, 0)
is, but we want to prioritise the second nation, which actually has an
army. 

\section{Buildings}

HoI naval bases are redistributed weighted by the Vic ones. Forts
convert directly except that the level is reduced by two, with coastal provinces gaining both sea and land
forts; the coast detection algorithm is heuristic
(specifically, it looks first for a naval base in the input, then in
the positions file to see if coordinates are given for a naval base)
and may miss some coastal provinces. 

Infrastructure converts ``urbanisation'', defined
as the ratio of clerks to labourers or farmers, plus he railroad value. The Victoria provinces
are sorted by urbanisation and assigned the input
infrastructure in order, so that the highest urbanisation
Victoria province gets the highest infrastructure in the
input. Capitals have their urbanity doubled. 

AA batteries convert like infrastructure, in the most urbanised
provinces, on the assumption that large cities get such protection. 

Air bases - vexed issue that they are - convert similarly to
infrastructure, but the weighting is the number of workers in airplane
factories, plus 100 times the naval-base level, plus 100 times the
urbanisation. In addition, every capital gets at least a level-1
airbase if it doesn't get one by other means. 

\section{Orders of Battle}

Land units are created in numbers equalling the vanilla setup, so that
each nation gets a number of HoI units proportional to the amount of the
corresponding Victoria units it has. For example, all four
kinds\footnote{And really, does any game need four kinds of cavalry?}
of Vicky cavalry (cavalry, dragoon, hussar, and cuirassier) correspond
to HoI cavalry. Consequently, if a nation has 25 Victoria cavalry
regiments (all kinds) and the total of such units in Victoria is 100,
then it gets HoI cavalry equal to one-fourth of the amount that exists
in the input save. The unit correspondences are listed in the
\texttt{unitTypes} object of the config file. Notice that reserve
units (from mobilisation) do not count as infantry; notice also that
not every HoI unit type has a corresponding Victoria one. 

In some cases additional units will be created. For example, the 1936
setup has only two armoured brigades (as opposed to light armour),
which is experentially a somewhat absurd constraint to impose on a
Victoria game in 1936. Consequently additional armoured brigades are
created in accordance with the \texttt{extraUnits} field:
\begin{verbatim}
extraUnits = {
  armor_brigade = { tank 5 10 15 20 25 35 50 75 100 125 150 175 200 225 250 300 350 400 450 500 600 }
}
\end{verbatim}
which says that if the world contains 5 tank units, an additional
armoured brigade is created, another at 10, and so on up to 600; after
600 there is one for every 100, the difference between the last and
the second-last entries. 

A sufficient amount of divisions, corps and higher formations, with headquarters, are created to
house the lower formations; so three (identical) brigades form a
division, three divisions (identical or not) form a corps, and so
on. Any formations at loose ends are attached to the single theatre
that is created for each nation. 

Ships are redistributed at random, weighted by the naval strength of
nations. Naval strength is defined as the sum of the weights given in
the \texttt{vicShips} field (that is, a dreadnought is 60, a cruiser
20, and so on), averaged with the naval support limit (which comes
from naval bases), \emph{unless} the former is higher than the latter,
in which case the naval force limit is used. Thus, suppose my naval
force limit is 100. If I build a single dreadnought (weight 60) my
naval strength is 80 (average of 60 and 100). If I build another
dreadnought (bringing the total weight to 120) my naval strength is
100 (the force limit). 

\section{Techs}

Most human players will be fully teched by 1936, so there is little to
distinguish nations on this point. The tech conversion is therefore
intended mainly to activate the obvious stuff, so players don't sit
about unable to build infantry divisions in 1936. The config file's
\texttt{techConversion} object contains fields of this form:
\begin{verbatim}
vicTech = { hoiTech hoiTech ... }
\end{verbatim}
where each \texttt{hoiTech} is increased to level one if the nation
has the \texttt{vicTech}. Otherwise all HoI techs start at zero. 

Practicals are gained from units, as regulated by the
\texttt{practicals} object. For example, the field
\begin{verbatim}
  infantry_practical = { infantry }
\end{verbatim}
indicates that Vicky infantry, as one might expect, gives the HoI
\texttt{infantry\_practical}. In particular, the nation with the most
Vicky infantry gains the highest practical that exists in the input
save; everyone else gets an amount proportional to their
infantry. Thus if the highest practical in historical 1936 is 10, and
Russia has 1000 infantry regiments in Victoria, a nation with 500
infantry regiments will get 5 infantry practical. For this purpose
forts and naval bases are weighted by level, and factories by the
number of employees. 

\section{Laws}

Law conversions are given by the \texttt{laws} object.
Laws convert in one of three ways:
\begin{itemize}
\item Points-based. The fields listed in \texttt{vicKeys} are
  examined, and the value gives the number of points listed in the
  \texttt{points} object. The total amount of points is then compared
  with the \texttt{hoiValues} object, and the law with the highest
  value less than the number of points is selected. 
\item Ratios. The \texttt{numerator} field is divided by
  \texttt{denominator} (note that many of these values do not exist in
  a Vicky save, but are calculated by the converter) and the selection
  then proceeds as for points - the law with the highest value less
  than the ratio is selected. 
\item By Victoria field: The \texttt{keyword} is examined and each
  possible value directly translates to a HoI law. 
\end{itemize}

\section{Miscellanea}

Manpower, officers, and resource stockpiles are distributed 
from the input, proportionally to the quantities listed in the
\texttt{stockpiles} object. The first entry indicates where in the HoI
country object the field is placed; the second, in what sub-object to
look for the Vicky quantity; the third and subsequent, the Vicky keywords. In both
cases \texttt{country} refers to the top-level nation
object. Note that for HoI, \texttt{cap\_pool} is not present in the
input save; it is constructed by the converter, and moved to the
nation's capital province after resources have been distributed. 

Wars, alliances, and vassalisations convert one for one; wargoals are
ignored. 

Factions are removed - no nation is a member of Axis, Allies, or
Comintern. 

Neutrality is linear in revanchism; it is 100 at 0 revanchism and
\texttt{minimumNeutrality} at full revanchism. 

Unity is linear in average militancy; it is 100 at 0 militancy and
\texttt{minimumUnity} at 10 militancy. 

Dissent is equal to the percentage of Vicky population that belongs to a
rebel faction. 

Terrain can only be modified by changing terrain.bmp, and the
converter therefore leaves it alone even though this means historical
placement of urban provinces. However, it does print out a list of the
100 provinces with the highest population of clerks, craftsmen, and
bureaucrats, for use in modding. Note that this list is only printed
if the debug stream labeled 50 in the config file is set to 'yes'. 

Victory points convert similarly to infrastructure: Victoria provinces
are weighted, and the highest weight receives the highest VP value in
the input save, and so down the list until there are no more input
VPs. In this case the weights are given by the \texttt{victoryPoints}
object in the config file; factory types are interpreted as the number
of workers in that kind of factory in the province. The
\texttt{navalBase} and \texttt{isCapital} fields are special; the
former is a multiplier for the force-limit contribution of any naval
base in the province, the second a multiplier applied to capitals. In addition,
all HoI provinces listed in \texttt{strategic\_provinces}, and all
capitals, get at least one victory point. 

\end{document}
